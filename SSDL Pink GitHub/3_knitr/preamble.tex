% ~~~~~~~~~~~~~~~~~~~~~~~~~~~~~~~~~~~~~~~~~~~~~~~~~~~~~~~~~~~~~~~~~~~~~~~~
%    1.  Language, Font
%    2.  Tables & Graphics
%    3.  Footnotes
%    4.  Citations 
%    5.  Layout
% ~~~~~~~~~~~~~~~~~~~~~~~~~~~~~~~~~~~~~~~~~~~~~~~~~~~~~~~~~~~~~~~~~~~~~~~~

% ~~~~~~~~~~~~~~~~~~~~~~~~~~~~~~~~~~~~~~~~~~~~~~~~~~~~~~~~~~~~~~~~~~~~~~~~
% 1.) Language, Font
% ~~~~~~~~~~~~~~~~~~~~~~~~~~~~~~~~~~~~~~~~~~~~~~~~~~~~~~~~~~~~~~~~~~~~~~~~

% Sprache
\usepackage[ngerman,english]{babel}
\addto{\captionsenglish}{\renewcommand{\bibname}{References}}

% T1 Schrift Encoding
\usepackage[T1]{fontenc}

% Kodierung
\usepackage[latin1]{inputenc} 		% Windows
\usepackage[right]{eurosym}				% Euro-Symbol verwendbar (durch \euro)

% ~~~~~~~~~~~~~~~~~~~~~~~~~~~~~~~~~~~~~~~~~~~~~~~~~~~~~~~~~~~~~~~~~~~~~~~~
% 2.) Tables & Graphics
% ~~~~~~~~~~~~~~~~~~~~~~~~~~~~~~~~~~~~~~~~~~~~~~~~~~~~~~~~~~~~~~~~~~~~~~~~
\usepackage{amsmath}
\usepackage{graphicx}
\usepackage{booktabs} 

\usepackage{multirow}
\usepackage{pdflscape}

\usepackage{tabularx}
\newcolumntype{C}{>{\centering\arraybackslash}X}
\newcolumntype{R}{>{\arraybackslash}X}
\newcolumntype{L}[1]{>{\centering\arraybackslash}p{#1}}
\usepackage[skip=0pt,labelfont=it,nooneline,small,justification=RaggedRight,singlelinecheck=off]{caption}

% Make figure captions look better
		\usepackage[format=plain,labelfont=bf]{caption}
		
% ~~~~~~~~~~~~~~~~~~~~~~~~~~~~~~~~~~~~~~~~~~~~~~~~~~~~~~~~~~~~~~~~~~~~~~~~
% 3.) Footnotes
% ~~~~~~~~~~~~~~~~~~~~~~~~~~~~~~~~~~~~~~~~~~~~~~~~~~~~~~~~~~~~~~~~~~~~~~~~
 
\usepackage[
   bottom,      % Footnotes appear always on bottom. This is necessary
                % especially when floats are used
   stable,      % Make footnotes stable in section titles
   ragged,      % Use RaggedRight
   multiple     % rearrange multiple footnotes intelligent in the text.
]{footmisc}

		\setlength{\footnotemargin}{-0.8em}
		\setlength{\dimen\footins}{10\baselineskip} % Beschraenkt den Platz von Fussnoten auf 10 Zeilen
		\interfootnotelinepenalty=10000 % Verhindert das Fortsetzen von Fussnoten auf der gegenüberligenden Seite


% ~~~~~~~~~~~~~~~~~~~~~~~~~~~~~~~~~~~~~~~~~~~~~~~~~~~~~~~~~~~~~~~~~~~~~~~~
% 4.) Citations
% ~~~~~~~~~~~~~~~~~~~~~~~~~~~~~~~~~~~~~~~~~~~~~~~~~~~~~~~~~~~~~~~~~~~~~~~~

% Zitation
\usepackage{natbib}		
\bibpunct[: ]{(}{)}{;}{a}{}{}
\setcitestyle{round,aysep={},yysep={,}}
 
% multiple authors
\usepackage[affil-it]{authblk} 

% ~~~~~~~~~~~~~~~~~~~~~~~~~~~~~~~~~~~~~~~~~~~~~~~~~~~~~~~~~~~~~~~~~~~~~~~~
% 5.) Layout
% ~~~~~~~~~~~~~~~~~~~~~~~~~~~~~~~~~~~~~~~~~~~~~~~~~~~~~~~~~~~~~~~~~~~~~~~~

\usepackage{lmodern}			% bessere Schriftdarstellung
\clubpenalty = 10000 			% reduziert Schusterjungen
\widowpenalty = 10000 			% reduziert Hurenkinder

% Optischer Randausgleich mit pdfTeX
\usepackage{microtype}

%Schöner aussehen im PDF
\usepackage{aeguill}
 
\setlength{\dimen\footins}{10\baselineskip} % Beschraenkt den Platz von Fussnoten auf 10 Zeilen
 
\interfootnotelinepenalty=10000 % Verhindert das Fortsetzen von
                                % Fussnoten auf der gegen�berligenden Seite

% Farbliche Verlinkung von Referenzen
\usepackage{color}	
\definecolor{darkblue}{rgb}{0,0,0.3}
\usepackage[citecolor=darkblue,linkcolor=darkblue,
urlcolor=black,colorlinks=true,bookmarksopen=true,
hyperfootnotes=false,pdftitle={title},plainpages=false,pdfpagelabels, %eindeutigere Zuordnung der Seiten
linktocpage=true,        % Inhaltsverzeichnis verlinkt Seiten
pdfauthor={Sebastian Pink},
pdfcreator={LaTeX}]{hyperref} 


\usepackage{setspace}
\onehalfspacing





